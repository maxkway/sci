\documentclass[a4paper,12pt]{article}

%%% Работа с русским языком
\usepackage{cmap}					% поиск в PDF
\usepackage{mathtext} 				% русские буквы в фомулах
\usepackage[T2A]{fontenc}			% кодировка
\usepackage[utf8]{inputenc}			% кодировка исходного текста
\usepackage[english,russian]{babel}	% локализация и переносы

%%% Дополнительная работа с математикой
\usepackage{amsmath,amsfonts,amssymb,amsthm,mathtools} % AMS
\usepackage{icomma} % "Умная" запятая: $0,2$ --- число, $0, 2$ --- перечисление

%% Номера формул
%\mathtoolsset{showonlyrefs=true} % Показывать номера только у тех формул, на которые есть \eqref{} в тексте.
%\usepackage{leqno} % Немуреация формул слева

%% Свои команды
\DeclareMathOperator{\sgn}{\mathop{sgn}}
\newcommand*\Laplace{\mathop{}\!\mathbin\bigtriangleup}

%% Перенос знаков в формулах (по Львовскому)
\newcommand*{\hm}[1]{#1\nobreak\discretionary{}
{\hbox{$\mathsurround=0pt #1$}}{}}

%%% Работа с картинками
\usepackage{graphicx}  % Для вставки рисунков
\graphicspath{{pic/}{images2/}}  % папки с картинками
\setlength\fboxsep{2pt} % Отступ рамки \fbox{} от рисунка
\setlength\fboxrule{1pt} % Толщина линий рамки \fbox{}
\usepackage{wrapfig} % Обтекание рисунков текстом




%%% Работа с таблицами
\usepackage{array,tabularx,tabulary,booktabs} % Дополнительная работа с таблицами
\usepackage{longtable}  % Длинные таблицы
\usepackage{multirow} % Слияние строк в таблице

%%% Теоремы
\theoremstyle{plain} % Это стиль по умолчанию, его можно не переопределять.
\newtheorem{theorem}{Теорема}[section]
\newtheorem{proposition}[theorem]{Утверждение}
 
\theoremstyle{definition} % "Определение"
\newtheorem{corollary}{Следствие}[theorem]
\newtheorem{problem}{Задача}[section]
 
\theoremstyle{remark} % "Примечание"
\newtheorem*{nonum}{Решение}

%%% Программирование
\usepackage{etoolbox} % логические операторы

%%% Страница
\usepackage{extsizes} % Возможность сделать 14-й шрифт
\usepackage{geometry} % Простой способ задавать поля
	\geometry{top=12mm}
	\geometry{bottom=12mm}
	\geometry{left=5mm}
	\geometry{right=5mm}
%
%\usepackage{fancyhdr} % Колонтитулы
% 	\pagestyle{fancy}
 	%\renewcommand{\headrulewidth}{0pt}  % Толщина линейки, отчеркивающей верхний колонтитул
% 	\lfoot{Нижний левый}
% 	\rfoot{Нижний правый}
% 	\rhead{Верхний правый}
% 	\chead{Верхний в центре}
% 	\lhead{Верхний левый}
%	\cfoot{Нижний в центре} % По умолчанию здесь номер страницы

\usepackage{setspace} % Интерлиньяж
%\onehalfspacing % Интерлиньяж 1.5
%\doublespacing % Интерлиньяж 2
%\singlespacing % Интерлиньяж 1

\usepackage{lastpage} % Узнать, сколько всего страниц в документе.

\usepackage{soul} % Модификаторы начертания
\usepackage{epigraph}
\usepackage{hyperref}
\usepackage[usenames,dvipsnames,svgnames,table,rgb]{xcolor}
\hypersetup{				% Гиперссылки
    unicode=true,           % русские буквы в раздела PDF
    pdftitle={hw},   % Заголовок
    pdfauthor={Шишкин Максим},      % Автор
    pdfsubject={Эффект Холла в полуповоднике},      % Тема
    pdfcreator={Создатель}, % Создатель
    pdfproducer={Производитель}, % Производитель
    pdfkeywords={hw}, % Ключевые слова
    colorlinks=true,       	% false: ссылки в рамках; true: цветные ссылки
    linkcolor=red,          % внутренние ссылки
    citecolor=black,        % на библиографию
    filecolor=magenta,      % на файлы
    urlcolor=cyan           % на URL
}

\usepackage{csquotes} % Инструменты для ссылок
\usepackage{cite} % Работа с библиографией
%\usepackage[superscript]{cite} % Ссылки в верхних индексах
%\usepackage[nocompress]{cite} % 

\usepackage{float} %force establishe parametr to figure H,h,!

\usepackage{multicol} % Несколько колонок

\usepackage{ mathrsfs }

\usepackage{physics}

\usepackage{mathtools}

\usepackage{empheq}
\usepackage{floatflt}
\usepackage[most]{tcolorbox}
\usepackage{enumerate}

%\usepackage{floatrow}


\usepackage{tikz}
\usetikzlibrary{fadings}
\usetikzlibrary{shadows.blur}
\usetikzlibrary{shapes}

\usepackage{nicematrix}
\NiceMatrixOptions{transparent,nullify-dots}

\usepackage{pgfplots}
\DeclareUnicodeCharacter{2212}{−}
\usepgfplotslibrary{groupplots,dateplot}
\usetikzlibrary{patterns,shapes.arrows}
\pgfplotsset{compat=newest}

\author{Максим Шишкин.}
\title{HW 1.}
\date{\today}

\begin{document}

\maketitle

\problem{Найти давление при изотермическом протекании по цилиндру вязкого газа.}
\nonum[Неправильное]{Некоторые рассуждения, приводящие к анзацу:
\begin{itemize}
    \item $p = \rho \frac{k_B T}{m}$ в силу уравнения Менделеева-Клайперона для газа.
    \item будем искать решение, при котором скорость направлена только по оси цилиндра $\mathbf{u} = u_z(z, r) \mathbf{e_z}$.\footnote{Зная, что это не так, сложно приводить аргументы в пользу этого, но по-простому это можно мотивировать тем, что это сложно представить.}
    \item Поскольку нет никого масштаба по z, при наличии закрепеленных границ(скорость на поверхности цилиндра 0), естественным кажется сохранения профиля, только с плавным перемасштабированием скорости по мере продвижения по цилиндру: $v_z = Z(z)R(r)$. (самоподобие течения в разных течениях)
    \item из уравнения неразрывности $\partial_z(\rho v_z) \implies u_z = \frac{q(r)}{\rho}$, что указывает на разделение переменных для плотности и давления соответственно.
\end{itemize}
Таким образом, мы приходим к анзацу $p = \tilde{p}(z)\Psi(r)$, $u_z = \frac{q(r)}{\Psi(r)}\frac{1}{\tilde{p}(z)}\frac{k_B T}{m}$
Выпишем стационарное уравнение Навье-Стокса для вязкой сжимающейся жидкости:
\begin{equation}\label{eq:ns}
    \rho(u_z\partial_z)u_z \mathbf{e_z} = - \nabla p + \eta \Laplace{ u_z} \mathbf{e_z} + \left(\zeta + \frac{\eta}{3}\right)\nabla (\partial_z u_z)
\end{equation}
\textit{Внезапный ход} - спроецируем на радиальную ось, подставив анзац выше:
\begin{equation}
    \tilde{p}(z)\partial_r\Psi(r) = \partial_r\left(\frac{q(r)}{\Psi(r)}\right)\partial_z\left(\frac{1}{\tilde{p}(z)}\right) \frac{k_B T}{m}\left(\zeta+\frac{\eta}{3}\right)
\end{equation}
Переменные тогда разделяются:
\begin{equation}
    \frac{1}{\tilde{p}(z)}\partial_z \left(\frac{1}{\tilde{p}(z)}\right) = \mathcal{C} = \frac{\partial_r \Psi(r)}{\partial_r(q/\Psi)} \frac{m}{k_B T}\frac{1}{\zeta + \frac{\eta}{3}}
\end{equation}
Отсюда находим, интегрируя:
\begin{equation}
    \tilde{p}(z) = \frac{1}{\sqrt{2\mathcal{C} z + c}}, \; \mathcal C \frac{q}{\Psi} = \frac{m}{k_B T}\frac{1}{\zeta+\eta/3}(\Psi + c_0)\implies u_z = \frac{1}{\mathcal{C}}\frac{1}{\zeta + \eta/3}(\Psi(r) +c_0)\sqrt{2z\mathcal{C}+c}
\end{equation}
Однако, теперь легко видеть, что в таком виде решение \textit{не подойдёт} для проекции \eqref{eq:ns} на ось $z$ - разные степени вхождения корня в слагаемых.
}
\newpage
\nonum[Более разумное]
Схема не сработала, так как предположение о нулевой радиальной скорости при ненулевом градиенте давления в радиальном направлении оказалось несовместимым с основым уравнением - уранвением на ось цилиндра. Или же утверждение о разделении переменных неверно.
Теперь будем действовать в рамках теории возмущения, считая что сжимаемость вносит лишь небольшие поправки:\footnote{Идеалогически схема будет похожа на адиабатическое приближение, когда <<эволюция>> по $z$ малая, так что эволюция по $r$ подстраивается при данной окрестности $z$.}
\begin{itemize}
    \item $\mathbf{u} = u_z\mathbf{e_z} + u_r\mathbf{e_r}$
    \item $u_r$ само по себе имеет некий порядок малости, причем её масштаб по $r\sim R$ так как в центре она ноль и на краях цилиндра, по $z$ она меняется слабо, поэтому дифферинцирование по $z$ вносит дополнительную малость.
    \item $v_z$ величина не малая, но её масштаб по $z$ - большой, так что её производные по $z$ малы, по $r$ масштаб тот же - $R$.
    \item градиет давления по $z$ не мал, однако следующие производные вносят малость, градиент же по $r$ мал.
\end{itemize}
Выпишем полное уравнение Навье-Стокса, указав, содержит каждый член малость порядка больше старшего в уравнении или нет.
\begin{equation} z-\text{axis: }
    \rho (u_z\textcolor{red}{\partial_z} + \textcolor{red}{u_r}\partial_r)u_z = -\partial_z p + \eta (\textcolor{red}{\Laplace_z} + \Laplace_r){u_z} + (\zeta + \eta/3)\textcolor{red}{\partial_z}(\partial_z u_z + u_r/r + \partial_r u_r)
\end{equation}
\begin{equation}\label{eq:r} r-\text{axis: }
     \rho (u_z\textcolor{red}{\partial_z} + \textcolor{red}{u_r}\partial_r)u_r = -\partial_r p + \eta (\textcolor{red}{\Laplace_z} + \Laplace_r - \frac{1}{r^2})u_r + 
     (\zeta + \eta/3)\partial_r(\underline{\partial_z u_z} + u_r/r + \partial_r u_r)
\end{equation}
Видно теперь, что, для оси $z$ получается в главном (не малом вовсе) уравнение как для несжимаемой жидкости:
\begin{equation}
    \partial_z p^{(-1)}\footnote{Индекс минус единица у давления подчеркивает, что само давленеи большое, а его производная порядка нулевого} = \eta \Laplace_r{} u_z^{(0)} \implies u_z^{0} = \frac{R^2-r^2}{4 \eta} (-\partial_z p^{(-1)})
\end{equation}
Кроме этого, мы имеет точный интеграл движения - поток через трубу постоянен во всех порядках теории возмущения, технически удобней весь поток удовлетворить в нулевом порядке, а в каждом следующей требовать равенства нулю потока в каждом следующем порядке теории возмущения.
\begin{equation}
    \int_S u_z^{(0)} \rho^{(0)} \dd{S} = q \pi R^2 \implies \rho^{(0)} (-\partial_z p^{(-1)}) R^2 = \textcolor{blue}{8}q\eta \implies \boxed{p^{(-1)} = \sqrt{\frac{\textcolor{blue}{16}q\eta}{R^2}\frac{k_B T}{m}}\sqrt{z_0-z}}
\end{equation}
\begin{equation}\boxed{
    v_z^{(0)} = \frac{R^2-r^2}{4\eta} \sqrt{\frac{\textcolor{blue}{16}q\eta}{R^2}\frac{k_B T}{m}}\frac{1}{2\sqrt{z_0-z}} 
}\end{equation}

Теперь самое время понять, что является малым параметром, для этохо необходимо сранить оставленные и отброшенные части:
\begin{eqnarray}
    \eta \frac{1}{R^2} u_z^{(0)} \; \text{    V.S.   } \rho \partial_z (u_z^{(0)})^{2} \;   \text{    V.S.   } \eta \partial_z^2 u_z^{(0)} \\
   \frac{\eta u_z}{R^2}  \; \text{    V.S.   } \frac{u_z \eta}{R^2}\, \frac{m u_z^2}{k_B T} \;   \text{    V.S.   } \frac{\eta u_z}{R^2} \left(\frac{\eta}{R\rho u_z}\right)^2 \left(\frac{m u_z^2 }{k_B T}\right)^2\\
    1 \; \text{    V.S.   } \frac{m u_z^2}{k_B T} \;   \text{    V.S.   } \left(\frac{\eta}{R\rho u_z}\right)^2 \left(\frac{m u_z^2 }{k_B T}\right)^2
\end{eqnarray}
Таким образом, направленное движение должно быть малым возмущением над тепловым:
\begin{equation}
    \frac{k_B T}{m u^2} \gg \text{max}\left(1, \frac{1}{\text{Re}}\right)
\end{equation}
Альтернативная запись второго условия: $k_B T \gg f_{\eta} \bar{l}$, т.е. энергия, связанная с тепловым движением молекулы должна быть больше работы сил вязкого трения на величине расстояния между молекулами.
Теперь, осозновая, что является малым параметром, мы можем сторить теорию возмущения дальше. 
В частности, из уравнения неразрывности, оказывается, что $v_r^{(1)} = 0$. Отсюда $\partial_r p^{(0)} = 0$.
Считая число рейнольдса не экстремальным, в первом порядке получаем уравнение:
\begin{equation}
    \partial_z p^{(0)}+ \rho^{(0)}u_z^{(0)}\partial_z u_z^{(0)} = \eta \Laplace_r u_z^{(1)}
\end{equation}
Содержательное задание - найти отклик именно на $u_z^{(0)}$
\begin{equation}
    \eta\Laplace_r u_z^{(1)} = - \sqrt{\frac{k_B T}{m}} \frac{1}{4(z_0-z)^{3/2}} \left(\frac{816q\eta}{R^2}\right)^{3/2} \left(\frac{R^2-r^2}{4\eta}\right)^2
\end{equation}
Ясно, что зависимость от $z$ останется той-же,что и у правой части, интегрирование же по $r$ производится явно. К этому нужно добавить решение, находящееся аналогично нулевому порядку $\sim \partial_z p^{(0)}$.
После посчитать потоки от каждого, помня, что плотность $\rho = \rho^{0} + \rho^{1}(= \frac{m}{k_B T} p^{(0)})$.
Равенство суммы потоков нулю зафиксирует $p^{(0)}$.
В общем, это уже чисто техническая задача..
Отметим лишь, что уже в следующем порядке у давления появится зависимость от радиуса, как и радиальная сокрость, что видно из наличия подчеркнутого члена в \eqref{eq:r}.
\end{document}
